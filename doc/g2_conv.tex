\documentclass[a4paper,12pt]{article}
\usepackage[utf8]{inputenc}
\usepackage[margin=0.7in]{geometry}
\usepackage{amsmath,amssymb}
\usepackage{caption}
\usepackage{verbatim}
\usepackage{hyperref}
\DeclareMathAlphabet{\mathpzc}{OT1}{pzc}{m}{it}

\newcommand{\aver}[1]{\ensuremath{\langle#1\rangle}}
\renewcommand{\t}{\ensuremath{\tau}}
\newcommand{\w}{\ensuremath{\omega}}
\newcommand{\W}{\ensuremath{\Omega}}
\newcommand{\n}{\ensuremath{\nu}}
\newcommand{\TT}{\ensuremath{\mathbb{T}_\t}}

\newcommand{\pom}{\ensuremath{\mathtt{Pomerol}}}

\begin{document}
\title{Relations between conventions for the two-particle Green's functions
       used in \texttt{Pomerol} and \texttt{TRIQS/CTHYB}}
\author{Igor Krivenko}
\maketitle

\texttt{Pomerol} definition of the two-particle Green's function:
\begin{equation}\label{g2_pomerol}
	G^{(2),\pom}_{\alpha\beta\gamma\delta}(\w_1,\w_2,\w_3) =
	- \int_0^\beta d\t_1d\t_1d\t_3
		e^{i\w_1\t_1 + i\w_2\t_2 - i\w_3\t_3}
		\aver{\TT
		c_\alpha(\t_1)c_\beta(\t_2)c^\dag_\gamma(\t_3)c^\dag_\delta(0)}.
\end{equation}

\texttt{TRIQS/CTHYB} definitions:
\begin{itemize}
	\item \textit{All fermionic case},
          \begin{equation}\label{g2_allfermionic}
	    G^{(2),3\nu}_{\alpha\beta\gamma\delta}(\nu_1,\nu_2,\nu_3) =
	    \int_0^\beta d\t_1d\t_1d\t_3
	    e^{-i\nu_1\t_1 + i\nu_2\t_2 - i\nu_3\t_3}
	    \aver{\TT
	      c^\dagger_\alpha(\t_1)c_\beta(\t_2)c^\dagger_\gamma(\t_3)c_\delta(0)}.
          \end{equation}
	\item \textit{Particle-hole channel},
		\begin{equation}\label{g2_ph}
			G^{(2),ph}_{\alpha\beta\gamma\delta}(\w;\n,\n') =
			\frac{1}{\beta}\int_0^\beta d\t_1d\t_2d\t_3d\t_4\
			e^{-i\n\t_1} e^{i(\n+\w)\tau_2} e^{-i(\n'+\w)\t_3} e^{i\n'\t_4}
			\aver{\TT
			c^\dag_\alpha(\t_1) c_\beta(\t_2) c^\dag_\gamma(\t_3)
			c_\delta(\t_4)}.
	\end{equation}

	\item \textit{Particle-particle channel},
		\begin{equation}\label{g2_pp}
			G^{(2),pp}_{\alpha\beta\gamma\delta}(\w;\n,\n') =
			\frac{1}{\beta}\int_0^\beta d\t_1d\t_2d\t_3d\t_4\
			e^{-i\n\t_1} e^{i(\w-\n')\t_2} e^{-i(\w-\n)\t_3} e^{i\n'\t_4}
			\aver{\TT
			c^\dag_\alpha(\t_1) c_\beta(\t_2) c^\dag_\gamma(\t_3)
			c_\delta(\t_4)}.
		\end{equation}
\end{itemize}

Relation between ph- and pp-channels:
\begin{equation}
	G^{(2),pp}_{\alpha\beta\gamma\delta}(\w;\n,\n') =
	G^{(2),ph}_{\alpha\beta\gamma\delta}(\w-\n-\n';\n,\n').
\end{equation}

General \texttt{Pomerol}-\texttt{TRIQS/CTHYB} relations disregarding block
structure.
\begin{align}
	G^{(2),3\nu}_{\alpha\beta\gamma\delta}(\nu_1,\nu_2,\nu_3) &=
	G^{(2),\pom}_{\beta\delta\alpha\gamma}(\nu_2,\nu_1+\nu_3-\nu_2,\nu_1),\\
	G^{(2),ph}_{\alpha\beta\gamma\delta}(\w;\n,\n') &=
	G^{(2),\pom}_{\beta\delta\alpha\gamma}(\w+\n,\n',\n),\\
	G^{(2),pp}_{\alpha\beta\gamma\delta}(\w;\n,\n') &=
	G^{(2),\pom}_{\beta\delta\alpha\gamma}(\w-\n',\n',\n).
\end{align}

Relations between $AABB$ and $ABBA$ block structures of $G^{(2)}$  in
\texttt{TRIQS/CTHYB}.
\begin{align}
G^{(2),ph}_{(A,a)(B,d)(B,c)(A,b)}(\w;\n,\n') &= -
G^{(2),ph}_{(A,a)(A,b)(B,c)(B,d)}(\n'-\n;\n,\n+\w),\\
G^{(2),pp}_{(A,a)(B,d)(B,c)(A,b)}(\w;\n,\n') &= -
G^{(2),pp}_{(A,a)(A,b)(B,c)(B,d)}(\w;\n,\w-\n').
\end{align}

Relations for $AABB$ block structure of $G^{(2)}$ in \texttt{TRIQS/CTHYB}.
\begin{align}
	G^{(2),ph}_{(A,a)(A,b)(B,c)(B,d)}(\w;\n,\n') &=
	G^{(2),\pom}_{(A,b)(B,d)(A,a)(B,c)}(\w+\n,\n',\n),\\
	G^{(2),pp}_{(A,a)(A,b)(B,c)(B,d)}(\w;\n,\n') &=
	G^{(2),\pom}_{(A,b)(B,d)(A,a)(B,c)}(\w-\n',\n',\n).
\end{align}

Relations for $ABBA$ block structure of $G^{(2)}$ in \texttt{TRIQS/CTHYB}.
\begin{align}
	G^{(2),ph}_{(A,a)(B,d)(B,c)(A,b)}(\w;\n,\n') &= -
	G^{(2),\pom}_{(A,b)(B,d)(A,a)(B,c)}(\n',\w+\n,\n),\\
	G^{(2),pp}_{(A,a)(B,d)(B,c)(A,b)}(\w;\n,\n') &= -
	G^{(2),\pom}_{(A,b)(B,d)(A,a)(B,c)}(\n',\w-\n',\n).
\end{align}

Mixed Matsubara/Legendre representation of $G^{(2)}$ as a transformation of
$G^{(2)}_{\alpha\beta\gamma\delta}(\w;\nu,\nu')$ (both channels)
\cite{LewinThesis}.
\begin{equation}\label{legendre_transform}
	G^{(2)}_{\alpha\beta\gamma\delta}(\w;\ell,\ell') \equiv
	\sum_{n,n'\in\mathbb{Z}}
	\bar T_{2n+m+1,\ell}
	G^{(2)}_{\alpha\beta\gamma\delta}(\w;\nu_n,\nu_{n'})
	\bar T^*_{2n'+m+1,\ell'},
\end{equation}
\begin{equation}
	\bar T_{o,\ell} \equiv \frac{\sqrt{2\ell+1}}{\beta}
	\int_0^\beta d\t e^{io\pi\frac{\t}{\beta}} P_\ell[x(\t)] =
	\sqrt{2\ell+1}i^o i^\ell j_\ell\left(\frac{o\pi}{2}\right).
\end{equation}

\bibliographystyle{plain}

\begin{thebibliography}{9}
	\bibitem{LewinThesis}
	Lewin Volker Boehnke,
	\emph{Susceptibilities in materials with multiple strongly correlated 
	orbitals}
	PhD thesis, Universit\"at Hamburg, 2015,
	\url{http://ediss.sub.uni-hamburg.de/volltexte/2015/7325/pdf/Dissertation.pdf}
\end{thebibliography}

\end{document}
